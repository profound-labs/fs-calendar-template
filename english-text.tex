
%% ===================
%% Pre-Layout Settings
%% ===================

\renewcommand\PreLayoutSettings{%
}% \PreLayoutSettings

%% ====================
%% Post-Layout Settings
%% ====================

\renewcommand\PostLayoutSettings{%

\renewcommand\calendarTitleText{Forest Sangha Calendar}

\renewcommand\branchMonasteries{BRANCH MONASTERIES}
\renewcommand\westernDisciplesOfAjahnChah{Western disciples of Ajahn Chah}
\renewcommand\portalPageWorldWide{The portal page for this community worldwide is:}

\renewcommand\noteTextAjChahMemorialDay{Ajahn Chah Memorial Day}
\renewcommand\noteTextAjChahBirthDay{Ajahn Chah's Birthday}
\renewcommand\noteTextVassaBegins{Vassa begins}

\renewcommand\textMaghaPuja{Māgha Pūjā}
\renewcommand\textVesakhaPuja{Vesākha Pūjā}
\renewcommand\textAsalhaPuja{Āsāḷhā Pūjā}
\renewcommand\textPavarana{Pavāraṇā}

\renewcommand\textUnitedKingdom{United Kingdom}
\renewcommand\textSwitzerland{Switzerland}
\renewcommand\textThailand{Thailand}
\renewcommand\textAustralia{Australia}
\renewcommand\textNewZealand{New Zealand}
\renewcommand\textUnitedStates{United States}
\renewcommand\textItaly{Italy}

}% \PostLayoutSettings

%% ===========
%% Frontmatter
%% ===========

\renewcommand\FrontMatterText{%

{\centering
\fontsize{10.8}{15}\selectfont
\setlength{\parskip}{0.7\baselineskip}

This calendar has been sponsored for free distribution\\
by the Kataññnutā group of Malaysia, Singapore and Australia.

Anumodanā to the many friends who have offered their photographs for this \theCalendarYear\ calendar,\\
in particular: Montri Sirithampiti (Apr., Nov.), Simone Anzini (cover, Jun., Aug.),\\
Boonchan Chanloung (Sep.), Andrew Binkley (Dec.), Chinch Gryniewicz (Feb.),\\
Gary Morrison (Oct.).

Monthly Dhamma quotes are adapted from translated teachings contained in\\
\emph{The Collected Teachings of Ajahn Chah}, available for download at the links below.

e-book formats:\hspace{0.3em} www.fsbooks.org/ajahn-chah-teachings

audiobook format:\hspace{0.3em} www.fsaudio.org/ajahn-chah-teachings

\vfill

{\large LUNAR OBSERVANCE DAYS
\hspace*{0.3em}
\tikz\node[new moon]{};
\hspace*{0.2em}
\begin{tikzpicture}
\node[waxing moon circle, name=waxing]{};
\node[waxing moon cover, above right=-1.6mm and 0pt of waxing.center]{};
\end{tikzpicture}
\hspace*{0.2em}
\tikz\node[full moon]{};
\hspace*{0.2em}
\begin{tikzpicture}
\node[waning moon circle, name=waning]{};
\node[waning moon cover, above left=-1.6mm and 0pt of waning.center]{};
\end{tikzpicture}
}

These days are regularly devoted to quiet reflection at the monastery.\\
The dates for the lunar calendar are determined by traditional methods of calculation\\
and are not always on the same day as the astronomical occurrences.

{\large THE MAJOR FULL MOON DAYS FOR \theCalendarYear\ / \theCalendarAltYear}

\emph{Māgha Pūjā} \spacedcdot\ 14 February (`Sangha Day')\\
Commemorates the spontaneous gathering of 1250 arahants to whom\\
the Buddha gave an exhortation on the basis of the Discipline (\emph{Ovāda Pāṭimokkha}).

\emph{Vesākha Pūjā} \spacedcdot\ 13 May (`Buddha Day')\\
Commemorates the birth, enlightenment and passing away of the Buddha.

\emph{Āsāḷhā Pūjā} \spacedcdot\ 11 July (`Dhamma Day')\\
Commemorates the Buddha's first discourse, given to the five \emph{samaṇas} in the Deer Park at\\
Sarnath, near Varanasi. The traditional Rainy-Season Retreat (\emph{Vassa}) begins on the next day.

\emph{Pavāraṇā Day} \spacedcdot\ 8 October\\
This marks the end of the three-month \emph{Vassa} retreat. During the following month,\\
lay people may offer the \emph{Kaṭhina} robe as part of a general alms-giving ceremony.

\vspace{0.3\baselineskip}

www.forestsangha.org\\
www.forestsanghapublications.org

Calendar production by Aruno Publications,\\
Aruna Ratanagiri Buddhist Monastery, UK\\
\copyright\ Aruno Publications \the\year\\
www.ratanagiri.org.uk

\vspace{0.9\baselineskip}

}

}% \FrontMatterText

%% ======
%% Quotes
%% ======

\renewcommand\SetAllQuotes{%

\SetQuoteText{1}{%
The essence of Buddhism is peace, and that peace arises\\
from truly knowing the nature of all things.
}

\SetQuoteText{2}{%
Once we are free, whatever our situation may be,\\
we won't have to suffer. If we have children, we won't have to suffer.\\
If we work, we won't have to suffer.
}

\SetQuoteText{3}{%
Practising generosity cleanses our hearts of selfishness;\\
our mind grows in compassion and caring towards all living beings.
}

\SetQuoteText{4}{%
There is one essential point that all good practice must eventually come to,\\
and that is not clinging. In the end,\\
all teachings and teachers must be let go of.
}

\SetQuoteText{5}{%
Everything you use in this life is support for the practice.\\
If your dwelling place is an utter mess then your mind will be the same.
}

\SetQuoteText{6}{%
We are taught first to abandon evil and establish that which is good.\\
Then we must transcend good and evil.
}

\SetQuoteText{7}{%
Walking the path, don’t be careless. Even if you are right, don't be careless.\\
If you are wrong, don't be careless.
}

\SetQuoteText{8}{%
This is the way practice should proceed:\\
firstly, we need to be upright and honest; secondly, to be wary of wrongdoing;\\
and thirdly, to have a heart imbued with humility.
}

\SetQuoteText{9}{%
Dhamma practice means upholding virtue, developing samādhi and\\
cultivating wisdom in our hearts. Reflect on the Triple Gem.\\
Strive on with sincerity.
}

\SetQuoteText{10}{%
Whatever virtues have been cultivated are imperfect if lacking in mindfulness.\\
Mindfulness is life. It is a cause for the arising\\
of self-awareness and wisdom.
}

\SetQuoteText{11}{%
Birth, ageing, illness, and death: these are universal truths.\\
See this clearly, acknowledge these facts\\
and you will be able to let go.
}

\SetQuoteText{12}{%
After awakening, the Buddha and his disciples still maintained their practice.\\
Effort was their way, their natural habit. I think we should take\\
their example as a model for our practice.
}

}% \SetAllQuotes

%% ==============
%% Photo Captions
%% ==============

\renewcommand\SetAllPhotoCaptions{%

\SetPhotoCaption{1}{%
Lake Tahoe, USA
}

\SetPhotoCaption{2}{%
Luang Por Sumedho giving a baby blessing, Wat Aruna Ratanagiri, UK
}

\SetPhotoCaption{3}{%
Ajahn Ñāṇarato, Nara Park, Japan
}

\SetPhotoCaption{4}{%
Ajahn Candasiri on alms-round (\textit{piṇḍapāta}), while visiting Wat Nanachat, Thailand
}

\SetPhotoCaption{5}{%
Ajahn Gavesako doing \textit{kuṭī} maintenance, Wat Cittaviveka, UK
}

\SetPhotoCaption{6}{%
Sāmaṇera Kovido receiving his alms-bowl during \textit{upasampadā}, Wat Santacittarama, Italy
}

\SetPhotoCaption{7}{%
Ajahn Khemasiri and friends, near Wat Dhammapala, Switzerland
}

\SetPhotoCaption{8}{%
Luang Por Boonchoo receiving offerings from Ajahn Chandapālo, Wat Santacittarama, Italy
}

\SetPhotoCaption{9}{%
Luang Por Ṭiradhammo on alms-round (\textit{piṇḍapāta}), Santaloka Hermitage, Italy
}

\SetPhotoCaption{10}{%
Kathina 2012, Wat Amaravati, UK
}

\SetPhotoCaption{11}{%
Funeral process for Luang Por, Tan Chao Khun Mahamon
}

\SetPhotoCaption{12}{%
Viewing rock carving, Thailand
}

}% \SetAllPhotoCaptions

%% End of text.

