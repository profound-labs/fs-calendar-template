
%% ===================
%% Pre-Layout Settings
%% ===================

\renewcommand\PreLayoutSettings{%
}% \PreLayoutSettings

%% ====================
%% Post-Layout Settings
%% ====================

\renewcommand\PostLayoutSettings{%

\renewcommand\calendarTitleText{Forest Sangha Calendar}

\renewcommand\branchMonasteries{BRANCH MONASTERIES}
\renewcommand\westernDisciplesOfAjahnChah{Western disciples of Ajahn Chah}
\renewcommand\portalPageWorldWide{The portal page for this community worldwide is:}

\renewcommand\noteTextAjChahMemorialDay{Ajahn Chah Memorial Day}
\renewcommand\noteTextAjChahBirthDay{Ajahn Chah's Birthday}
\renewcommand\noteTextVassaBegins{Vassa begins}

\renewcommand\textMaghaPuja{Māgha Pūjā}
\renewcommand\textVesakhaPuja{Vesākha Pūjā}
\renewcommand\textAsalhaPuja{Āsāḷhā Pūjā}
\renewcommand\textPavarana{Pavāraṇā}

\renewcommand\textUnitedKingdom{United Kingdom}
\renewcommand\textSwitzerland{Switzerland}
\renewcommand\textThailand{Thailand}
\renewcommand\textAustralia{Australia}
\renewcommand\textNewZealand{New Zealand}
\renewcommand\textUnitedStates{United States}
\renewcommand\textItaly{Italy}
\renewcommand\textCanada{Canada}

}% \PostLayoutSettings

%% ===========
%% Frontmatter
%% ===========

\renewcommand\FrontMatterText{%

{\centering
\frontMatterFmt

This calendar has been sponsored for free distribution\\
by the Kataññnutā group of Malaysia, Singapore and Australia.

% TODO: Add credits
Many friends and supporters generously made their photographs available for Sangha publications.\\
This \theCalendarYear\ calendar includes:
Person~Name (Apr., Nov.),
Person~Name (cover, Jun., Aug.),
Person~Name (Sep.),
Person~Name (Dec.),
Person~Name (Feb.),
Person~Name (Oct.).
and several others of whose names we are not aware. \emph{Anumodanā}.

% TODO: cover description
Cover: `\emph{Asking for forgiveness}.' Detail from a temple mural\\
painted by Khun Pang Chinasai, Aruna Ratanagiri Buddhist Monastery, UK

Monthly Dhamma quotes are adapted from transcribed and translated teachings contained in\\
\emph{The Collected Teachings of Ajahn Chah}, available for download at the links below.

e-book formats:\hspace{0.3em} www.fsbooks.org/ajahn-chah-teachings

audiobook format:\hspace{0.3em} www.fsaudio.org/ajahn-chah-teachings

\vfill

{\large LUNAR OBSERVANCE DAYS
\hspace*{0.3em}
\tikz\node[new moon]{};
\hspace*{0.2em}
\begin{tikzpicture}
\node[waxing moon circle, name=waxing]{};
\node[waxing moon cover, above right=-1.6mm and 0pt of waxing.center]{};
\end{tikzpicture}
\hspace*{0.2em}
\tikz\node[full moon]{};
\hspace*{0.2em}
\begin{tikzpicture}
\node[waning moon circle, name=waning]{};
\node[waning moon cover, above left=-1.6mm and 0pt of waning.center]{};
\end{tikzpicture}
}

These days are regularly devoted to quiet reflection at the monastery.\\
The dates for the lunar calendar are determined by traditional methods of calculation\\
and are not always on the same day as the astronomical occurrences.

{\large THE MAJOR FULL MOON DAYS FOR \theCalendarYear\ / \theCalendarAltYear}

% TODO: Add date
\emph{Māgha Pūjā} \spacedcdot\ XX February (`Sangha Day')\\
Commemorates the spontaneous gathering of 1250 arahants to whom\\
the Buddha gave an exhortation on the basis of the Discipline (\emph{Ovāda Pāṭimokkha}).

% TODO: Add date
\emph{Vesākha Pūjā} \spacedcdot\ XX May (`Buddha Day')\\
Commemorates the birth, enlightenment and passing away of the Buddha.

% TODO: Add date
\emph{Āsāḷhā Pūjā} \spacedcdot\ XX July (`Dhamma Day')\\
Commemorates the Buddha's first discourse, given to the five \emph{samaṇas} in the Deer Park at\\
Sarnath, near Varanasi. The traditional Rainy-Season Retreat (\emph{Vassa}) begins on the next day.

% TODO: Add date
\emph{Pavāraṇā Day} \spacedcdot\ XX October\\
This marks the end of the three-month \emph{Vassa} retreat. During the following month,\\
lay people may offer the \emph{Kaṭhina} robe as part of a general alms-giving ceremony.

\vspace{0.3\baselineskip}

www.forestsangha.org\\
www.forestsanghapublications.org

% TODO: Copyright date is \theCalendarYear - 1
Calendar production by Aruno Publications,\\
Aruna Ratanagiri Buddhist Monastery, UK

\copyright\ Aruno Publications \the\year\\
www.ratanagiri.org.uk

\vspace{0.9\baselineskip}

}

}% \FrontMatterText

%% ======
%% Quotes
%% ======

\renewcommand\SetAllQuotes{%

% TODO: add quote
\SetQuoteText{1}{%
Lorem ipsum dolor sit amet, consectetuer\\
adipiscing elit. Ut purus elit, vestibulum ut, placerat ac,\\
adipiscing vitae, felis. Curabitur dictum gravida mauris.
}

% TODO: add quote
\SetQuoteText{2}{%
Lorem ipsum dolor sit amet, consectetuer\\
adipiscing elit. Ut purus elit, vestibulum ut, placerat ac,\\
adipiscing vitae, felis. Curabitur dictum gravida mauris.
}

% TODO: add quote
\SetQuoteText{3}{%
Nam arcu libero, nonummy eget, consectetuer id, vulputate a, magna.\\
Donec vehicula augue eu neque. Pellentesque habitant morbi tristique\\
senectus et netus et malesuada fames ac turpis egestas.
}

% TODO: add quote
\SetQuoteText{4}{%
Mauris ut leo. Cras viverra metus rhoncus sem.\\
Nulla et lectus vestibulum urna fringilla ultrices.
}

% TODO: add quote
\SetQuoteText{5}{%
Integer sapien est, iaculis in, pretium quis, viverra ac, nunc.\\
Praesent eget sem vel leo ultrices bibendum. Aenean faucibus.
}

% TODO: add quote
\SetQuoteText{6}{%
Morbi dolor nulla, malesuada eu, pulvinar at, mollis ac, nulla.\\
Curabitur auctor semper nulla. Donec varius orci eget risus.\\
Duis nibh mi, congue eu, accumsan eleifend, sagittis quis, diam.
}

% TODO: add quote
\SetQuoteText{7}{%
Duis eget orci sit amet orci dignissim rutrum.\\
Nam dui ligula, fringilla a, euismod sodales, sollicitudin vel, wisi.\\
Morbi auctor lorem non justo.
}

% TODO: add quote
\SetQuoteText{8}{%
Nam lacus libero, pretium at, lobortis vitae, ultricies et, tellus.\\
Donec aliquet, tortor sed accumsan bibendum, erat ligula aliquet magna,\\
vitae ornare odio metus a mi. Morbi ac orci et nisl hendrerit mollis.
}

% TODO: add quote
\SetQuoteText{9}{%
Suspendisse ut massa.\\
Cras nec ante.\\
Pellentesque a nulla.
}

% TODO: add quote
\SetQuoteText{10}{%
Cum sociis natoque penatibus et magnis dis parturient montes,\\
nascetur ridiculus mus. Aliquam tincidunt urna.\\
Nulla ullamcorper vestibulum turpis.
}

% TODO: add quote
\SetQuoteText{11}{%
Pellentesque cursus luctus mauris. Nulla malesuada porttitor diam.\\
Donec felis erat, congue non, volutpat at, tincidunt tristique, libero.
}

% TODO: add quote
\SetQuoteText{12}{%
Vivamus viverra fermentum felis. Donec nonummy pellentesque ante.\\
Phasellus adipiscing semper elit. Proin fermentum massa ac quam.\\
Sed diam turpis, molestie vitae, placerat a, molestie nec, leo.
}

}% \SetAllQuotes

%% ==============
%% Photo Captions
%% ==============

\renewcommand\SetAllPhotoCaptions{%

% TODO: add caption
\SetPhotoCaption{1}{%
Vivamus viverra fermentum felis
}

% TODO: add caption
\SetPhotoCaption{2}{%
Donec nonummy pellentesque ante
}

% TODO: add caption
\SetPhotoCaption{3}{%
Phasellus adipiscing semper elit
}

% TODO: add caption
\SetPhotoCaption{4}{%
Proin fermentum massa ac quam
}

% TODO: add caption
\SetPhotoCaption{5}{%
Sed diam turpis, molestie vitae, placerat a, molestie nec, leo
}

% TODO: add caption
\SetPhotoCaption{6}{%
Maecenas lacinia
}

% TODO: add caption
\SetPhotoCaption{7}{%
Nam ipsum ligula, eleifend at, accumsan nec, suscipit a, ipsum
}

% TODO: add caption
\SetPhotoCaption{8}{%
Morbi blandit ligula feugiat magna
}

% TODO: add caption
\SetPhotoCaption{9}{%
Nunc eleifend consequat lorem
}

% TODO: add caption
\SetPhotoCaption{10}{%
Sed lacinia nulla vitae enim
}

% TODO: add caption
\SetPhotoCaption{11}{%
Pellentesque tincidunt purus vel magna
}

% TODO: add caption
\SetPhotoCaption{12}{%
Integer non enim
}

}% \SetAllPhotoCaptions

%% End of text.

